%iffalse
\let\negmedspace\undefined
\let\negthickspace\undefined
\documentclass[journal,12pt,twocolumn]{IEEEtran}
\usepackage{cite}
\usepackage{amsmath,amssymb,amsfonts,amsthm}
\usepackage{algorithmic}
\usepackage{graphicx}
\usepackage{textcomp}
\usepackage{xcolor}
\usepackage{txfonts}
\usepackage{listings}
\usepackage{enumitem}
\usepackage{mathtools}
\usepackage{gensymb}
\usepackage{comment}
\usepackage[breaklinks=true]{hyperref}
\usepackage{tkz-euclide} 
\usepackage{listings}
\usepackage{gvv}                                        
%\def\inputGnumericTable{}                                 
\usepackage[latin1]{inputenc}                                
\usepackage{color}                                            
\usepackage{array}                                            
\usepackage{longtable}                                       
\usepackage{calc}                                             
\usepackage{multirow}                                         
\usepackage{hhline}                                           
\usepackage{ifthen}                                           
\usepackage{lscape}
\usepackage{tabularx}
\usepackage{array}
\usepackage{float}


\newtheorem{theorem}{Theorem}[section]
\newtheorem{problem}{Problem}
\newtheorem{proposition}{Proposition}[section]
\newtheorem{lemma}{Lemma}[section]
\newtheorem{corollary}[theorem]{Corollary}
\newtheorem{example}{Example}[section]
\newtheorem{definition}[problem]{Definition}
\newcommand{\BEQA}{\begin{eqnarray}}
\newcommand{\EEQA}{\end{eqnarray}}
\newcommand{\define}{\stackrel{\triangle}{=}}
\theoremstyle{remark}
\newtheorem{rem}{Remark}

% Marks the beginning of the document
\begin{document}
\bibliographystyle{IEEEtran}
\vspace{3cm}

\title{16.APPLICATIONS OF DERIVATIVES}
\author{EE24BTECH11058 - P.SHINY DIAVAJNA}

\maketitle
\newpage
\bigskip


\renewcommand{\thefigure}{\theenumi}
\renewcommand{\thetable}{\theenumi}

 \begin{enumerate}
  \item[\textbf{I.} ]\textbf{Section-A JEE Advanced/ IIT-JEE}\\
 \end{enumerate} 
   \begin{enumerate}
  \item[\textbf{C.}] \textbf{MCQs with One Correct Answer}\\
   \end{enumerate}
    \begin{enumerate} 
     \item[\textbf{24.}]If $f(x)=x^3+bx^2+cx+d$ and $0<b^2<c,$ then in $(-\infty,\infty)$ \hfill{\textbf{(2004S)}}
     \begin{enumerate}
         \item[(a)] $f(x)$ is a strictly increasing function
         \item[(b)] $f(x)$ has a local maxima
         \item[(c)] $f(x)$ is a strictly decreasing function
         \item[(d)] $f(x)$ is bounded  \\
     \end{enumerate}
    \end{enumerate}
 \begin{enumerate}
    \item[\textbf{25.}] If $f(x)=x^{\alpha} logx$ and $f(0)=0,$ then the value of $\alpha$ for which Rolles's theorem can be applied in [0,1] is 
    \hfill{\textbf{(2004S)}} \\
    
    (a) -2 \\
    (b) -1 \\(c) 0 \\ (d) 1/2 \\
    
    \item[\textbf{26.}] If $P(x)$ is a polynomial of degree less than or equal to 2 and $S$ is the set of all such polynomials so that $P(0)=0,P(1)=1$ and $P'(x)>0$ $\forall x \in [0,1],$ then
    \hfill{\textbf{(2005S)}} 
    \begin{enumerate}
        \item [(a)] $S=\phi$
        \item [(b)] $S=ax+(1-a)x^2$ $\forall$ a $\in (0,2)$
        \item [(c)] $S=ax+(1-a)x^2$ $\forall$ a $\in (0,\infty)$
        \item [(d)] $S=ax+(1-a)x^2$ $\forall$ a $\in (0,1)$ \\
    \end{enumerate} 
   \end{enumerate} 
   
   \begin{enumerate}
      \item [\textbf{27.}] The tangent to the curve $y=e^x$ drawn at the point $\vec{(c,e^c)}$intersects the line joining the points $\vec{(c-1,e^{c-1})}$ and $\vec{(c+1,e^{c+1})}$
    
      \hfill {\textbf{(2007-3 marks)}}\\
      (a) on the left of $x$=c \\
      (b)on the right of$x$=c \\
      (c) at no point \\  
      (d)at all points \\


      
    \end{enumerate}
    \begin{enumerate}
    \item[\textbf{28.}]Consider the two curves $C_{1}:y^2=4x,$$C_{2}:x^2+y^2-6x+1=0.)$ Then,  \hfill{\textbf{(2008)}}
    
    (a) $C_{1}$ and $C_{2}$ touch each other only at one point.
    
    (b) $C_{1}$ and $C_{2}$ touch each other exactly at two points

    (c) $C_{1}$ and $C_{2}$ intersect (but do not touch) at exactly two points

    (d) $C_{1}$ and $C_{2}$ neither intersect nor touch each other 
     
  \end{enumerate}
  \begin{enumerate}
   \item[\textbf{29.} ]The total number of local maxima and local minima of the function \\
   
  $ f(x)=\begin{cases} 
   (2+x)^3, -3<x\le -1\\
   x^{2/3} ,-1<x<2
   \end{cases}$ 3 is
   \hfill{\textbf{(2008)}}

   (a) 0 \\
   (b) 1\\
   (c) 2 \\ 
   (d) 3 \\
       
\end{enumerate}
\begin{enumerate}
   \item[\textbf{30.}]  Let the function $g:(-\infty,\infty) \rightarrow  \brak{-\frac{\pi}{2},\frac{\pi}{2}}$ be given by $g(u)= 2 tan^{-1}\brak{e^u}-\frac{\pi}{2}$. Then,g is  
   
   \hfill{\textbf{(2008)}}
   \begin{enumerate}

   \item[(a)] even and is strictly increasing in (0,$\infty$)
 
   \item[(b)] odd and is strictly decreasing in ($-\infty,\infty$)


   \item[(c)] odd and is strictly increasing in ($-\infty,\infty$)

   \item [(d)]neither even nor odd,but is strictly increasing in ($-\infty,\infty$) \\

\end{enumerate}    
\end{enumerate}
 \begin{enumerate}
     \item [\textbf{31.}] The least value of a $\in$ $\mathbb{R}$ for which $4\alpha x^2 + \frac{1}{x} \ge 1,$ for all $x>0,$ is 
     \hfill{\textbf{(JEE Adv. 2016)}}\\
     (a) $\frac{1}{64}$\\
     (b) $\frac{1}{32}$\\
     (c) $\frac{1}{27}$\\
     (d) $\frac{1}{25}$\\
     \item[\textbf{32.}] If $f: R \rightarrow R$ is a twice differentiable function such that $f''(x)>0$ for all $x \in R$ and $f\brak{\frac{1}{2}} = \brak{\frac{1}{2}},$ $f(1)=1,$ then
     \hfill{\textbf{(JEE Adv. 2017)}}\\
     (a) $f'(1) \le 0$ \\
     (b) $0<f'(1) \le \frac{1}{2}$\\
     (c) $\frac{1}{2} < f'(1) \le 1$ \\
     (d) $f'(1) > 1$
     
 \end{enumerate}
 \newpage
 \textbf{D. MCQs With One or More than One Correct}\\
 \begin{enumerate}
     \item [\textbf{1.}] Let $P(x) = a_0+ a_1x^2+a_2x^4+......a_nx^{2n}$ be a polynomial in a real variable x with \\
     $0<a_0<a_1<a_2<.....a_n.$ The function $P(x)$ has 
     \hfill{\textbf{(1986- 2 Marks)}}
      \begin{enumerate}

   \item[(a)] neither a maximum nor a minimum
 
   \item[(b)] only one maximum

   \item[(c)] only one minimum

   \item [(d)] only one maximum and only one minimum
   
   \item[(e)] none of these.\\

\end{enumerate} 

\item[\textbf{2.}] If the line $ax+by+c = 0$ is a normal to the curve $xy=1,$ then 
\hfill{\textbf{(1986-2 Marks)}}\\
 (a)$a>0,b>0$ \\
 (b)$a>0,b<0$ \\
 (c)$a<0,b>0$\\
 (d)$a<0,b<0$\\
 (e) none of these.\\

\item[\textbf{3.} ] The smallest positive root of the equation, $tanx-x=0$ lies in 
\hfill{\textbf{(1987-2 Marks)}}\\
 (a) $\brak{0,\frac{\pi}{2}}$\\
 (b) $\brak{\frac{\pi}{2},\pi}$\\
 (c) $\brak{\pi,\frac{3\pi}{2}}$\\
 (d) $\brak{\frac{3\pi}{2},2\pi}$\\
 (e) None of these\\

 \item[\textbf{4.}] Let $f$ and $g$ be increasing and decreasing functions, respectively from $[0,\infty)$ to $[0,\infty)$. Let $h(x) = f(g(x).$ If $h(0) = 0,$ then $h(x)-h(1)$ is
 \hfill{\textbf{(1987-2 Marks)}}\\
 (a) always zero\\
 (b) always negative\\
 (c) always positive\\
 (d) strictly increasing\\
 (e) None of these.\\

 \item[\textbf{5.}] If  
 $ f(x)=\begin{cases} 
   3x^2+12x-1,-1 \le x\le 2\\
   37-x ,2<x \le 3
   \end{cases}$ then:
   
   \hfill{\textbf{(2008)}}

   (a) $f(x)$ is increasing on [-1,2]\\
   (b) $f(x)$ is continuous on [-1,3]\\
   (c) $f'(2)$ does not exist\\
   (d) $f(x)$ has the maximum value at $x=2$\\

\item[\textbf{6.}] Let $h(x) = f(x)-(f(x))^2+ (f(x))^3$  for every real number $x.$ Then
\hfill{\textbf{(1998-2 Marks)}}\\
   (a) $h$ is increasing whenever $f$ is increasing\\
   (b) $h$ is increasing whenever $f$ is decreasing\\
   (c) $h$ is decreasing whenever $f$ is decreasing\\
   (d) nothing can be said in general.\\


   

 
 
 

 
 

     
 \end{enumerate}
 

 






\end{document}
